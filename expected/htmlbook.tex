\documentclass[a5paper,onecolumn]{book}

\usepackage[utf8]{inputenc}
\usepackage[T1]{fontenc}
\usepackage[hidelinks]{hyperref}
\usepackage{graphicx}

\usepackage[top=3.3cm, bottom=3.3cm, left=2cm, right=2cm]{geometry}
\newif\ifpdf
\ifx\pdfoutput\undefined
  \pdffalse
\else
  \ifnum\pdfoutput=1
    \pdftrue
  \else
    \pdffalse
  \fi
\fi

\title{HTMLBook}
\author{}
\date{}

\newcounter{sectionnr}

\begin{document}

\maketitle

\clearpage

\thispagestyle{empty}

\pagestyle{empty}
\subsection*{\begin{center} \textbf{Introduction} \end{center}}


 \noindent
 This is an introduction. This example gamebook is for testing the htmlbook option. See https://github.com/oreillymedia/HTMLBook for more information about HTMLBook. 
\vspace{1em}
\subsection*{\begin{center} \textbf{More} \end{center}}


 \noindent
 This is another introduction section, just to see that formatting is correct when there are more than one. 
\vspace{1em}

Turn to 1 to begin.
\phantomsection
\refstepcounter{sectionnr}
\label{section1}
\subsection*{\begin{center} \textbf{1} \end{center}}

 \noindent
 Got to this section. Story ends here.
\vspace{1em}
\phantomsection
\refstepcounter{sectionnr}
\label{section2}
\subsection*{\begin{center} \textbf{2} \end{center}}

 \noindent
 Got to another section. Story ends here. 
\vspace{1em}
\phantomsection
\refstepcounter{sectionnr}
\label{section3}
\subsection*{\begin{center} \textbf{3} \end{center}}

 \noindent
 This is the first section. From here you can go to the unnamed section \textbf{\autoref{section2}} or this named section (\textbf{\autoref{section1}})
. 
\vspace{1em}
\end{document}
