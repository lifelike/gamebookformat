\documentclass[a5paper,onecolumn]{book}

\usepackage[utf8]{inputenc}
\usepackage[T1]{fontenc}
\usepackage[hidelinks]{hyperref}
\usepackage{graphicx}

\usepackage[top=3.3cm, bottom=3.3cm, left=2cm, right=2cm]{geometry}
\newif\ifpdf
\ifx\pdfoutput\undefined
  \pdffalse
\else
  \ifnum\pdfoutput=1
    \pdftrue
  \else
    \pdffalse
  \fi
\fi

\title{Paged Option}
\author{}
\date{}

\newcounter{sectionnr}

\begin{document}

\maketitle

\clearpage

\thispagestyle{empty}

\pagestyle{empty}
\subsection*{\begin{center} \textbf{Introduction} \end{center}}


 \noindent
 This gamebook is formatted with the paged option. Every section should begin at the top of a new page. Not the introsections though. 
\vspace{1em}
\subsection*{\begin{center} \textbf{Another Heading} \end{center}}


 \noindent
 This is another introsection, just to confirm that it is not cause a new page-break. 
\vspace{1em}
\subsection*{\begin{center} \textbf{Formats} \end{center}}


 \noindent
 Only some formats support page-breaks. HTML for instance currently does not (although it is technically possible to add it using print stylesheets). 
\vspace{1em}

Turn to 1 to begin.
\clearpage
\phantomsection
\refstepcounter{sectionnr}
\label{section1}
\subsection*{\begin{center} \textbf{1} \end{center}}

 \noindent
 First section. You can go to \textbf{\autoref{section4}} or \textbf{\autoref{section3}} from here. 
\vspace{1em}
\clearpage
\phantomsection
\refstepcounter{sectionnr}
\label{section2}
\subsection*{\begin{center} \textbf{2} \end{center}}

 \noindent
 This is The End. 
\vspace{1em}
\clearpage
\phantomsection
\refstepcounter{sectionnr}
\label{section3}
\subsection*{\begin{center} \textbf{3} \end{center}}

 \noindent
 This is the third section. It might or might not be numbered 3 though, because of the automatic section numbering in gamebookformat. You can go on to the end at \textbf{\autoref{section2}}. 
\vspace{1em}
\clearpage
\phantomsection
\refstepcounter{sectionnr}
\label{section4}
\subsection*{\begin{center} \textbf{4} \end{center}}

 \noindent
 This is the second section. Yay. You can go on to \textbf{\autoref{section3}} or \textbf{\autoref{section2}} or back to \textbf{\autoref{section1}}. 
\vspace{1em}
\end{document}
