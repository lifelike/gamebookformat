\documentclass[a5paper,onecolumn]{book}

\usepackage[utf8]{inputenc}
\usepackage[T1]{fontenc}
\usepackage[hidelinks]{hyperref}
\usepackage{graphicx}

\usepackage[top=3.3cm, bottom=3.3cm, left=2cm, right=2cm]{geometry}
\newif\ifpdf
\ifx\pdfoutput\undefined
  \pdffalse
\else
  \ifnum\pdfoutput=1
    \pdftrue
  \else
    \pdffalse
  \fi
\fi

\title{Format}
\author{}
\date{}

\newcounter{sectionnr}

\begin{document}

\maketitle

\thispagestyle{empty}

\pagestyle{empty}

\clearpage

\subsection*{\begin{center} \textbf{Introduction} \end{center}}


 \noindent
 Adding an introduction to the gamebook here. This will create a section, but it will not be shuffled nor numbered with the gamebook sections below. 
\vspace{1em}
\subsection*{\begin{center} \textbf{Another Heading} \end{center}}


 \noindent
 This starts another non-shuffled section. 
\vspace{1em}


Adventure begins in section 1.
\phantomsection
\refstepcounter{sectionnr}
\label{section1}
\subsection*{\begin{center} \textbf{1} \end{center}}

 \noindent
 This examples tests gamebook formatting, not so much game mechanics or references. Currently there is nothing here really. This section contains some tricky characters to quote, like \} and \{ and " and ' and \textbackslash. HTML will probably not like <div> or \&boom;. There should be an image below as well. If something broke, turn to \textbf{\autoref{section2}}, otherwise turn to \textbf{\autoref{section3}}. \begin{center}
\includegraphics[width=.9\textwidth]{testimage.png}
\end{center}
 
 
\vspace{1em}
\phantomsection
\refstepcounter{sectionnr}
\label{section2}
\subsection*{\begin{center} \textbf{2} \end{center}}

 \noindent
 Bad.
\vspace{1em}
\phantomsection
\refstepcounter{sectionnr}
\label{section3}
\subsection*{\begin{center} \textbf{3} \end{center}}

 \noindent
 Good! 
\vspace{1em}
\refstepcounter{sectionnr}
\end{document}
